\documentclass{article}%
\usepackage{cfr-lm}%
\usepackage[T1]{fontenc}%
% \usepackage[utf8]{inputenc}%
\usepackage{physics}
\usepackage{amsmath}
\usepackage{amssymb}
\usepackage{graphicx}
\usepackage[margin=3cm]{geometry}
\usepackage{changepage}
\usepackage{fontspec}
\usepackage{xcolor}
% \usepackage{tikz}
%
%
\title{Oblig 1}%
\author{Brage Wiseth}%
\date{\today}%
\newfontfamily{\iosevka}{Iosevka Fixed Curly Semibold Extended}%
\newcommand{\æ}{\textunderscore}
\newcommand{\å}{\iosevka}
%
\begin{document}%
\normalsize%
\iosevka
{\let\newpage\relax\maketitle}
\normalfont
\vspace*{.8cm}
% \hspace*{-5.4mm}%
%
%
%
\section*{Oppgave 2}
For å implementere "Teque" bruker jeg to sirkulere arrays {\iosevka f} og {\iosevka b} som representerer hver sin halvdel av køen. Hver array har en tilhørende {\iosevka front} og en {\iosevka back} peker.
De døpes {\iosevka f\textunderscore front, f\textunderscore back, b\textunderscore front, b\textunderscore back}. {\iosevka f} for første halvdel {\iosevka b} for andre halvdel. 
{\iosevka b\textunderscore front} fortsetter køen der {\iosevka f\textunderscore back} slapp.
Det vil si at hvis {\iosevka b\textunderscore front} tilsvarer indeks $5$ så vil {\iosevka f\textunderscore back} tilsvare indeks $4$.
{\iosevka f} og {\iosevka b} har hver sin referanse til mengder elementer i seg selv. Det brukes til å balansere de to.   \\\\
    \begin{minipage}[t]{.48\linewidth}
        \subsection*{Algorithm: Push front} 
        {\sbweight Input:} Ett element {\iosevka x}\\
        {\sbweight Output:} Plasserer {\iosevka x} fremst i køen\\\\
        \iosevka
        \color{darkgray}
        1 procedure push\textunderscore front(x) \\
        2\hspace*{6mm} f[f\textunderscore front] $\leftarrow$  x\\
        3\hspace*{6mm} f\textunderscore front++ and f\æ size++\\
        4\hspace*{6mm} if f\textunderscore front > |f|-1 then\\
        5\hspace*{12mm} f\textunderscore front $\leftarrow$ 0 \\
        6\hspace*{6mm} balanser()
    \end{minipage}
    \hspace{4mm}
    \begin{minipage}[t]{.48\linewidth}
        \subsection*{Algorithm: Push back} 
        {\sbweight Input:} Ett element {\å x}\\
        {\sbweight Output:} Plasserer {\iosevka x} sist i køen\\\\
        \iosevka
        \color{darkgray}
        1 procedure push\textunderscore back(x) \\
        2\hspace*{6mm} b[b\textunderscore back] $\leftarrow$ x\\
        3\hspace*{6mm} b\textunderscore back-- and b\æ size++\\
        4\hspace*{6mm} if b\textunderscore back < 0 then \\
        5\hspace*{12mm} b\textunderscore back $\leftarrow$ |b|-1 \\
        6\hspace*{6mm} balanser()  
    \end{minipage}\\\\\\\\
    \begin{minipage}[t]{.48\linewidth}
        \subsection*{Algorithm: push middle} 
        {\sbweight Input:} Ett element {\å x}\\
        {\sbweight Output:} Plasserer {\iosevka x} midt i køen\\\\
        \iosevka
        \color{darkgray}
        1 procedure push\textunderscore middle(x) \\
        2\hspace*{6mm} if f\æ size < b\æ size then \\
        3\hspace*{12mm} f[f\æ back] $\leftarrow$ b[b\æ front-1]\\
        4\hspace*{12mm} b[b\æ front-1] $\leftarrow$ x \\
        5\hspace*{12mm} f\æ back-- and f\æ size++\\
        6\hspace*{12mm} if f\æ back < 0 then\\
        7\hspace*{18mm} f\æ back $\leftarrow$ |f|-1\\
        8\hspace*{6mm} else then\\
        9\hspace*{12mm} b[b\æ front] $\leftarrow$ x\\
        10\hspace*{10mm} b\æ front++ and b\æ size++\\
        11\hspace*{10mm} if b\æ front > |b|-1 then\\
        12\hspace*{16mm} b\æ front $\leftarrow$ 0 \\
    \end{minipage}
    \hspace{4mm}
    \begin{minipage}[t]{.48\linewidth}
        \subsection*{Algorithm: get} 
        {\sbweight Input:} En indeks {\å i}\\
        {\sbweight Output:} element på indeks {\å i}\\\\
        \iosevka
        \color{darkgray}
        1 procedure get(i) \\
        2\hspace*{6mm} if i < f\æ size then\\
        3\hspace*{12mm} j $\leftarrow$ (f\æ front-1) - i\\
        4\hspace*{12mm} if j < 0 then\\
        5\hspace*{18mm} return f[|f| + j]\\
        6\hspace*{12mm} else return f[j]\\
        7\hspace*{6mm} else then\\
        8\hspace*{12mm} j $\leftarrow$ b\æ (front-1) - (i-f\æ size)\\
        9\hspace*{12mm} if j < 0 then\\
        10\hspace*{16mm} return b[|b| + j]\\
        11\hspace*{10mm} else return b[j]\\
    \end{minipage}\\\\
Til slutt har vi hjelpemetoden {\å balanser()}:\\
\begin{center}
    \begin{minipage}[t]{.48\linewidth}
        \subsection*{Algorithm: balanser} 
        \iosevka
        \color{darkgray}
        1 procedure balanser() \\
        2\hspace*{6mm} d $\leftarrow$ f\æ size - b\æ size\\
        3\hspace*{6mm} if d > 1 then\\
        4\hspace*{12mm} f\æ back++\\
        5\hspace*{12mm} b[b\æ front] $\leftarrow$ f[f\æ back]\\
        6\hspace*{12mm} f[f\æ back] $\leftarrow$ null\\
        7\hspace*{12mm} b\æ front++\\
        8\hspace*{12mm} if b\æ front > |b|-1 then\\
        9\hspace*{18mm} b\æ front $\leftarrow$ 0\\
        10\hspace*{10mm} b\æ size++ and f\æ size--\\
        11\hspace*{4mm} if d < -1 then\\
        12\hspace*{10mm} b\æ front-- \\
        13\hspace*{10mm} f[f\æ back] $\leftarrow$ b[b\æ front]\\
        14\hspace*{10mm} b[b\æ front] $\leftarrow$ null\\
        15\hspace*{10mm} f\æ back--\\
        16\hspace*{10mm} if f\æ back < 0 then \\
        17\hspace*{16mm} f\æ back $\leftarrow$ |f|-1 \\
        18\hspace*{10mm} b\æ size-- and f\æ size++\\
    \end{minipage}
\end{center}
    %
%
%
\end{document}%     